\preamble{Các Loại Cụm Từ Trong Tiếng Anh}{Các Loại Cụm Từ Trong Tiếng Anh}

\chapter{Định Nghĩa}
Một cụm từ là một hoặc nhiều từ kết hợp để cấu thành một đơn vị có nghĩa trong một mệnh đề.\\
Có tất cả 3 loại cụm từ:
\begin{enumerate}
    \item Cụm danh từ
    \item Cụm danh từ
    \item Cụm tính từ
\end{enumerate}

\chapter{Các loại cụm từ}
\section{Noun Phrase (Cụm danh từ)}
\rhead{\Large Cụm Danh Từ}
Cụm danh từ có thể là một danh từ hoặc một nhóm từ xây dựng xung quanh một danh từ\\
\textbf{Ví dụ:} a beautiful trip, the last sandwich\\\\
Cụm Danh từ trong tiếng Anh thường được cấu tạo bởi công thức sau:\\
Hạn định từ + bổ ngữ đứng trước + Danh từ chính + bổ ngữ đứng sau\\\\
Hạn định từ bao gồm:
\begin{enumerate}
    \item Mạo từ (article): a/ an/ the
    \item Đại từ chỉ định (demonstrative Pronouns): this/ that/ those/ these
    \item Từ chỉ số lượng / Số thứ tự (Quantifiers): four, three, third, second,…
    \item Tính từ sở hữu: his, her,…
    \item Bổ ngữ đứng trước thường là tính từ (adjectives)
\end{enumerate}
Bổ ngữ đứng sau thường là cụm giới từ hoặc một mệnh đề.


\section{Phrasal Verb (Cụm động từ)}
\rhead{Cụm Động Từ}
Cụm động từ được tạo thành từ sự kết hợp giữa một động từ và một từ nhỏ (particle). Particle(s), này có thể là một trạng từ (adverb), hay là một giới từ (preposition), hoặc là cả hai. 
\\Khi đó cụm động từ sẽ có nghĩa khác hoàn toàn với động từ vốn có, vì vậy hãy coi chúng như một từ mới hoàn toàn.
\\Ví dụ: look up, look after, get on with\\\\
\textbf{3 dạng cụm động từ phổ biến:}
\begin{enumerate}
    \item Verb + Adverb
    \item Verb + Preposition
    \item Verb + Adverb + Preposition
\end{enumerate}

\section{Phrasal Adjective (Cụm tính từ)}
\rhead{Cụm tính từ}
Thực chất là một tính từ đi kèm một giới từ, thể hiện một nghĩa nhất định nào đó.\\
Ví dụ: interested in, fond of, willing to