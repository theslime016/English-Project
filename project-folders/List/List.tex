\rhead{Main Knowledge}
\title{\Huge \textbf{Danh Sách}}
\maketitle
\tableofcontents

\chapter{12 Thì}
\chapter{Từ Vựng}
\begin{multicols}{2}
\begin{itemize}
    \item Danh từ
	\item Động từ
	\item Danh động từ
	\item Động từ to be
	\item Động từ nguyên thể
	\item Động từ chính
	\item Động từ khuyết thiếu
	\item Động từ nhận thức, giác quan
	\item Tính từ
	\item Trạng từ
	\item Mạo từ
	\item Đại từ
	\item Giới từ
	\item Lượng Từ
	\item Từ nối
	\item Tân ngữ
	\item Sở Hữu Cách
	\item Tiền tố và hậu tố
	\item Quy tắc thêm -s/-es/-ies, -ing và -ed
	\item Sự hòa hợp giữa chủ ngữ và động từ
	\item Cách đọc số
\end{itemize}
\end{multicols}

\chapter{Câu, Mệnh Đề Và Thể}
\begin{multicols}{2}
\begin{itemize}
    \item Cấu trúc ngữ pháp của một câu
    \item Câu đơn, câu ghép
    \item Câu Phức, câu phức tổng hợp
    \item Câu trần thuật
    \item Câu nghi vấn
    \item Câu cầu khiến
    \item Câu mệnh lệnh
    \item Câu điều kiện
    \item Câu mong ước
    \item Câu hỏi đuôi
    \item Câu hỏi với từ để hỏi
    \item Câu hỏi/Câu trả lời ngắn
    \item Mệnh đề
    \item Sự khác nhau giữa mệnh đề với câu   
    \item và cụm từ
    \item Mệnh đề độc lập/phụ thuộc
    \item Mệnh đề trạng ngữ
    \item Mệnh đề tính ngữ
    \item Mệnh đề danh ngữ
    \item Mệnh đề -ing/-ed
    \item Mệnh đề điều kiện
    \item Đảo ngữ
    \item Thể của động từ
    \item Thể bị động
    \item Mệnh lệnh cách
    \item Bàng thái cách
\end{itemize}
\end{multicols}

\chapter{Phát Âm}
\begin{multicols}{2}
\begin{itemize}
    \item Bảng IPA
    \item Trọng âm và cách đánh trọng âm
    \item Cách nối âm và nuốt âm
    \item Ngữ Điệu
    \item Cách phát âm -s/-es, -ed và các âm cuối khác
    \item Âm câm và các âm câm phổ biến
\end{itemize}
\end{multicols}

\chapter{Các Cấu Trúc Câu Phổ Biến}
\chapter{Cách Sử Dụng Các Từ Phổ Biến}
\end{document}