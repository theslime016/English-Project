\preamble{Infinitives In English}{Infinitives}
\chapter{Definition}
Infinitives are a form of verb that allow the word or a group of words to be used as a noun, adjective, or adverb. 
Every type of verb can be put into the infinitive form, even phrasal verbs.\\\\
Usually, infinitives are formed by adding the word to before the base form of the verb, as in to be, 
but sometimes the base form of the verb is used alone (we explain more in the next section).\\\\
The purpose of infinitives is to discuss an action in general instead of a specific instance of the action being done. 
For example, take a look at these two sentences:
\begin{enumerate}
    \item[] I need \textbf{to win}. 
    \item[] Today, we \textbf{win}. 
\end{enumerate}



\chapter{Types}
There are two main types of infinitives: \textbf{full infinitives} and \textbf{bare infinitives}.
\vspace{1cm}
\section{Full infinitives(To infinitives)}
Full infinitives, also known as to-infinitives, are the most common infinitives in writing. 
You can create a full infinitive by taking the base form of a verb and adding to in front of it.\\\\
Full infinitives are used in the following situations:



\subsection{To show purpose or intention}
Infinitives are used to explain why someone is doing something, often replacing the phrase “in order to.”\\\\
\begin{enumerate}
    \item[] Mom left \textbf{to buy} milk.  
    \item[] I’m writing this email \textbf{to tell} you something important.
    \item[] Did you come to college \textbf{to study} or \textbf{to party}? 
\end{enumerate}
Unlike an adverbial clause, an infinitive phrase used as an adverb does not need an active verb. 



\subsection{To modify nouns}
Just like how full infinitives can add extra information about verbs, 
they can also modify the meanings of nouns. In this case, 
they act as adjectives and adjective phrases. \\\\
\begin{enumerate}
    \item[] We need a hero \textbf{to save} us. 
    \item[] Would you like something \textbf{to drink}? 
    \item[] It was a dumb thing \textbf{to say}, and I regret it. 
\end{enumerate}




\subsection{As the subject of a sentence}
If you want to talk about an action in general as the subject of the sentence, 
use the full infinitive form. \\\\
\begin{enumerate}
    \item[] \textbf{To love} someone requires patience and understanding. 
    \item[] \textbf{To go} this late seems pointless. 
    \item[] \textbf{To unlearn} is the highest form of learning. 
\end{enumerate}





\subsection{After adjectives}
Full infinitives can add context or extra description when used after adjectives. \\\\
\begin{enumerate}
    \item[] I’m happy \textbf{to be} here. 
    \item[] Isn’t it nice \textbf{to leave} the city? 
    \item[] Computers are easy \textbf{to use} with practice. 
\end{enumerate}





\subsection{With the words too or enough}
When using the adverbs too and enough, we use the full infinitive to explain why. 
In these cases, the infinitive is often unnecessary, 
but it’s nonetheless a helpful addition if the sentence is vague. \\\\
\begin{enumerate}
    \item[] I have too many books \textbf{to fit} in my backpack. 
    \item[] We collected enough firewood \textbf{to last} the winter. 
    \item[] They were old enough \textbf{to vote} but not \textbf{to drink}. 
\end{enumerate}




\subsection{Phrases with most relative pronouns}
Use the full infinitive in phrases that start with one of the relative pronouns 
who, whom, what, where, when, and how—but not why.\\\\ 
\begin{enumerate}
    \item[] I don’t understand how \textbf{to beat} the Level 5 boss. 
    \item[] Playing cards is about knowing when \textbf{to hold} them and when \textbf{to fold} them. 
    \item[] Here’s a list of whom \textbf{to call} in an emergency. 
\end{enumerate}
Keep in mind that you only use full infinitives when relative pronouns are used as phrases, 
but not typically when used for questions:
\begin{enumerate}
    \item[] I don’t know what to do.
    \item[] What to do?
\end{enumerate}



\subsection{With certain verbs}
Certain verbs always use the full infinitive if they’re followed by a verb form. 
These words can still be used without an infinitive at all—but if they use an infinitive, 
it should be the full infinitive. \\\\
\begin{itemize}
    \item \textit{afford}
    \item \textit{agree}
    \item \textit{aim}
    \item \textit{appear}
    \item \textit{arrange}
    \item \textit{attempt}
    \item \textit{beg}
    \item \textit{care}
    \item \textit{etc.}
\end{itemize}






\section{Bare infinitives(Zero infinitives)}
Bare infinitives, also known as zero infinitives, 
are formed without to—you simply use the base form of a verb within a sentence.\\\\
Bare infinitives are used in the following situations:


\subsection{After modal verbs}
When using an infinitive after modal verbs, you don’t need to include to. 
Common modal verbs include: \textit{can}, \textit{may}, \textit{might}, \textit{could}, 
\textit{should}, \textit{would}, \textit{will}, and \textit{must}.
\begin{enumerate}
    \item[] Iggy can \textbf{do} this all day. 
    \item[] We might \textbf{be} late tonight. 
    \item[] You must not \textbf{mention} politics when talking to my father. 
\end{enumerate}





\subsection{After perception verbs}

Perception verbs (\textit{see}, \textit{ear}, \textit{taste}, \textit{feel}, etc.) 
use bare infinitives when their object takes an action. \\
In this case, the order is \textbf{main verb → object → bare infinitive}:
\begin{enumerate}
    \item[] I heard the car \textbf{arrive} before I saw it. 
    \item[] They felt the ants \textbf{crawl} on their arm. 
    \item[] She watched the woman in the red dress \textbf{walk} across the dance floor. 
\end{enumerate}






\subsection{With the verbs let, make, and do}
Just like certain verbs always use the full infinitive, 
a few verbs always use the bare infinitive. 
These include the common verbs \textit{let}, \textit{make}, and \textit{do}. 
Keep in mind the verbs let and make often use a direct object, 
which comes between them and the bare infinitive. 
\begin{itemize}
    \item[] Let me \textbf{work} in peace!
    \item[] He made him \textbf{promise} to behave. 
    \item[] I don’t \textbf{drink} coffee in the evening. 
\end{itemize}



\subsection{With the relative pronoun why}
While the other relative pronouns use the full infinitive form, 
the word why uses the bare infinitive, especially when used to make suggestions in the form of a question. 
\begin{itemize}
    \item[] Why \textbf{wear} a raincoat when it’s sunny outside? 
    \item[] Why not \textbf{ask} for directions? 
    \item[] Why \textbf{bother}?
\end{itemize}