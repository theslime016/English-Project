\rhead{Main Knowledge}
\title{\Huge \textbf{Danh Sách}}
\maketitle
\tableofcontents

\chapter{12 Thì}
\href{https://github.com/theslime016/english-project/blob/main/english-project-release-files/12Tenses.pdf}
{.pdf file}
\chapter{Từ Vựng}
\begin{multicols*}{2}
    \begin{itemize}
        \item \href{https://github.com/theslime016/english-project/blob/main/english-project-release-files/Noun.pdf}{Danh từ}
        \item \href{https://github.com/theslime016/english-project/blob/main/english-project-release-files/Verb.pdf}{Động từ}
        \item \href{https://github.com/theslime016/english-project/blob/main/english-project-release-files/Phrasal.pdf}{Cụm động từ, cụm động từ và cụm tính từ}
        \item \href{https://github.com/theslime016/english-project/blob/main/english-project-release-files/Gerunds.pdf}{Danh động từ}
        \item \href{https://github.com/theslime016/english-project/blob/main/english-project-release-files/To%20be%20Verbs.pdf}{Động từ to be}
        \item Động từ nguyên thể
        \item Động từ chính
        \item Động từ khuyết thiếu
        \item Động từ nhận thức, giác quan
        \item Tính từ
        \item Trạng từ
        \item Mạo từ
        \item Đại từ
        \item Giới từ
        \item Lượng Từ
        \item Từ nối
        \item Tân ngữ
        \item Sở Hữu Cách
        \item Tiền tố và hậu tố
        \item Quy tắc thêm -s/-es/-ies, -ing và -ed
        \item Sự hòa hợp giữa chủ ngữ và động từ
        \item Cách đọc số
    \end{itemize}
\end{multicols*}

\chapter{Câu, Mệnh Đề Và Thể}
\begin{multicols*}{2}
    \begin{itemize}
        \item Cấu trúc ngữ pháp của một câu
        \item Câu đơn, câu ghép
        \item Câu Phức, câu phức tổng hợp
        \item Câu trần thuật
        \item Câu nghi vấn
        \item Câu cầu khiến
        \item Câu mệnh lệnh
        \item Câu điều kiện
        \item Câu mong ước
        \item Câu hỏi đuôi
        \item Câu hỏi với từ để hỏi
        \item Câu hỏi/Câu trả lời ngắn
        \item Mệnh đề
        \item Sự khác nhau giữa mệnh đề với câu và cụm từ
        \item Mệnh đề độc lập/phụ thuộc
        \item Mệnh đề trạng ngữ
        \item Mệnh đề tính ngữ
        \item Mệnh đề danh ngữ
        \item Mệnh đề -ing/-ed
        \item Mệnh đề điều kiện
        \item Đảo ngữ
        \item Thể của động từ
        \item Thể bị động
        \item Mệnh lệnh cách
        \item Bàng thái cách
    \end{itemize}
\end{multicols*}

\chapter{Phát Âm}
\begin{multicols*}{2}
    \begin{itemize}
        \item \href{https://www.cambridge.org/features/IPAchart/}{Bảng IPA}
        \item Trọng âm và cách đánh trọng âm
        \item Cách nối âm và nuốt âm
        \item Ngữ Điệu
        \item Cách phát âm -s/-es, -ed và các âm cuối khác
        \item Âm câm và các âm câm phổ biến
    \end{itemize}
\end{multicols*}

\chapter{Các Cấu Trúc Câu Phổ Biến}
\begin{multicols*}{2}
    \begin{itemize}
        \item Collocation và các collocation phổ biến
        \item Các dạng so sánh
        \item Quy tắc biến đổi tính từ và trạng từ trong So sánh hơn nhất
        \item The same as…
        \item More and more
        \item The more…, the more…
        \item As well as
        \item Not only… but also
        \item So
        \item Enough
        \item Prefer
        \item What a
        \item Tương lai gần:
        \begin{itemize}
            \item S + will + V
            \item S + BE + V\udsc{ing}
            \item BE going to + V
            \item Be about to + V
            \item BE to + V
        \end{itemize}
        \item[]
        \item Câu đề nghị:
        \begin{itemize}
            \item Would you like
            \item Suggest
            \item Let’s
            \item What/How about
            \item Why not/don’t…?
            \item Do/Would you mind..?
            \item Can/Could…, please?
            \item Can/Shall I…?
        \end{itemize}
    \end{itemize}
\end{multicols*}
\chapter{Cách Sử Dụng Các Từ Phổ Biến}
\begin{multicols*}{2}
    \begin{itemize}
        \item There
        \item Used to
        \item Mind
        \item Hope
        \item Until
        \item Yet
        \item Just
        \item Own
        \item Such
        \item How much \& How many
        \item Must, Have \& Should
        \item Can, May \& BE able to
        \item House \& Home
        \item Take, Get \& Give
        \item Day \& Date
        \item Who, Whom \& Whose
        \item What, Which \& Why
        \item Older \& Elder
        \item Also \& Too
        \item It \& One
        \item The, A \& An
        \item Some \& Any
        \item No, Nothing \& None
        \item Other \& Another
        \item Each other \& One another
        \item Do, BE \& have
        \item Can \& Could
        \item Many \& Might
        \item Ought to, Dare \& Need
        \item As if \& As though
        \item Enter, Go, Walk, Get \& Come
        \item Look, Seem \& Appear
        \item Photo \& Picture
        \item In, At \& On
        \item Introduce \& Present
        \item Inform \& Announce
        \item Few \& A few
        \item Little \& A Little
        \item Learn \& Study
        \item When, While \& As
        \item Be \& Become
        \item Talk, Tell \& Speak
        \item So, Very \& Too
        \item Kind \& Type
        \item Last \& Final
        \item Profile \& Records        
    \end{itemize}
\end{multicols*}

\end{document}